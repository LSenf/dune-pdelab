\documentclass[11pt,a4paper,DIV11,%draft,%
notitlepage,oneside,abstracton,%
bibtotoc]{scrartcl}

%encoding
\usepackage[utf8]{inputenc}

% only for the article version
\usepackage{amsmath}
\usepackage{amsfonts}
\usepackage{amsthm}
\usepackage{fullpage}
%\setlength{\parindent}{0pt}
%\setlength{\parskip}{1.3ex plus 0.5ex minus 0.2ex}


% all after
\usepackage{graphicx}
%\usepackage{multimedia}
\usepackage{psfrag}
\usepackage{listings}
\lstset{language=C++, basicstyle=\small\ttfamily,
  stringstyle=\ttfamily, commentstyle=\it, extendedchars=true}
\usepackage{curves}
\usepackage{epic}
\usepackage{calc}
\usepackage{fancybox}
\usepackage{xspace}
\usepackage{enumerate}
\usepackage{algorithmic}
\usepackage{algorithm}
\usepackage{hyperref}

\title{Multi-Step Methods}
\author{\textsc{Jö Fahlke}\\
Interdisziplinäres Zentrum für Wissenschaftliches Rechnen,\\
 Universität Heidelberg, Im Neuenheimer Feld 368, \\
D-69120 Heidelberg\\
email: \texttt{jorrit@jorrit.de}
}
\date{Version: \today}

\begin{document}

\maketitle

%\begin{abstract}
%\end{abstract}

\tableofcontents

\section{Framework}

Consider the $p$'th order ODE
\begin{equation}
  f(t,u,\partial_tu,\partial_t^2u,\ldots,\partial_t^pu)=0.
\end{equation}
Often, it can be written as a sum of residuals
\begin{equation}
  \sum_{j=0}^pr_j(t,\partial_t^ju)=0
\end{equation}
We consider $m$-step schemes ($m\geq p$) that can be written in the following
way to solve for $u^n$:
\begin{equation}
  \sum_{i=0}^m\sum_{j=0}^p{\alpha_{ij} \over (\Delta t)^j}r_j(t^{n-i},u^{n-i})=0
\end{equation}

\section{Particular Schemes}

\subsection{Central Differences}

We consider the vector wave equation from Jin, a second order ODE:
\begin{equation}
  T\partial_t^2u+T_\sigma\partial_tu+Su+f=0
\end{equation}
Jin gives the central differences time stepping scheme (a two-step scheme) as
follows:
\begin{equation}
  \left\{{1 \over (\Delta t)^2} T + {1 \over 2\Delta t} T_\sigma \right\} u^n
  +\left\{-{2 \over (\Delta t)^2} T + S \right\} u^{n-1}
  +\left\{{1 \over (\Delta t)^2} T + {1 \over 2\Delta t} T_\sigma \right\} u^{n-2}
  +f^{n-1} = 0
\end{equation}
We can set the residuals as follows:
\begin{align}
  r_0(t,x)&=S(t)x+f(t) \\
  r_1(t,x)&=T_\sigma(t)x \\
  r_2(t,x)&=T(t)x
\end{align}
The coefficients $\alpha_{ij}$ then have to be
\begin{align}
  \alpha_{00}&=0 & \alpha_{01}&=\frac12 & \alpha_{02}&=1 \\
  \alpha_{10}&=1 & \alpha_{11}&=0       & \alpha_{12}&=-2 \\
  \alpha_{20}&=0 & \alpha_{21}&=\frac12 & \alpha_{22}&=1
\end{align}

\subsection{Newmark $\beta$ Method}

We consider the same ODE as for central differences.  The residuals are also
the same.  Jin gives the Newmark-$\beta$ scheme as follows:
\begin{multline}
  \left\{{1 \over (\Delta t)^2}T + {1 \over 2\Delta t}T_\sigma + \beta S \right\} u^n
  +\left\{-{2 \over (\Delta t)^2}T + (1-2\beta)S \right\} u^{n-1} \\
  +\left\{{1 \over (\Delta t)^2}T + {1 \over 2\Delta t}T_\sigma + \beta S \right\} u^{n-2}
  +[\beta f^n + (1-2\beta)f^{n-1} + \beta f^{n-2}] = 0
\end{multline}
The coefficients $\alpha_{ij}$ then have to be
\begin{align}
  \alpha_{00}&=\beta    & \alpha_{01}&=\frac12 & \alpha_{02}&=1 \\
  \alpha_{10}&=1-2\beta & \alpha_{11}&=0       & \alpha_{12}&=-2 \\
  \alpha_{20}&=\beta    & \alpha_{21}&=\frac12 & \alpha_{22}&=1
\end{align}
Central differences is identical to Newmark-$\beta$ with $\beta=0$!

\subsection{BDF}

We consider an $m$-step BDF method as given by Wikipedia
$<$\url{http://en.wikipedia.org/wiki/Backward_differentiation_formula}$>$:
\begin{equation}
  {1 \over \Delta t} \sum_{i=0}^m a_i u^{n-i} = b_0 f(t^n,u^n)
\end{equation}
for the first oder ODE
\begin{equation}
  \partial_tu = f(t,u)
\end{equation}
We chose the residuals
\begin{align}
  r_0(t,x)&=f(t,x) \\
  r_1(t,x)&=x
\end{align}
and the coefficients $\alpha_{ij}$:
\begin{equation}\begin{aligned}
  \alpha_{00}&=b_0 \\
  \alpha_{i0}&=0    && 1 \leq i \leq m \\
  \alpha_{i1}&=-a_i && 0 \leq i \leq m
\end{aligned}\end{equation}

\section{Implementation}

\subsection{\tt MultiStepParameters}

Parameter class providing
\begin{enumerate}
\item the number of stages $m$,
\item the order of the ODE $p \leq m$, and
\item the coefficients $\alpha_{ij}$, $0 \leq i \leq m$, $0 \leq j \leq p$.
\end{enumerate}

\subsection{\tt MultiStepGridOperatorSpace}

{\tt GridOperatorSpace} taking a {\tt MultiStepParameters} object and $p+1$
local operators for the residuals $r_0,\ldots,r_p$.
\begin{enumerate}
\item $\Delta t$ and current time $t^n$ are stored internally.
\item (Private) method {\tt residualAtStep()} takes a step number $i>0$ and a
  vector of dofs $u^{n-i}$ and computes
  \begin{equation}
    \sum_{j=0}^p{\alpha_{ij} \over (\Delta t)^j}r_j(t^{n-i},u^{n-i})
  \end{equation}
\item {\tt setTime()} updates the current time $t^n$.
\item Method {\tt preStep()} receives the dof vectors $u^{n-m},\ldots,u^{n-1}$
  to compute the constant part of the residual
  \begin{equation}
    \sum_{i=1}^m\sum_{j=0}^p{\alpha_{ij} \over (\Delta t)^j}r_j(t^{n-i},u^{n-i})
  \end{equation}
  It must be called before calling {\tt residual()} but after {\tt setTime()}.
\item The {\tt MultiStepGridOperatorSpace} calls {\tt setTime()} on the local
  operators as necessary.  The local operators should call {\tt setTime()} on
  any parameter functions they evaluate, as apropriate.
\item Initially the {\tt MultiStepGridOperatorSpace} may be limited to a
  certain ODE order $p$, because handling an arbitrary number of local
  operators of probably different C++ types will be involved.
\end{enumerate}

\subsection{\tt MultiStepMethod}

This class takes a {\tt MultiStepGridOperatorSpace} object, (possibly) a {\tt
  MultiStepParameters} object and a solver object (linear or newton) and feed
the {\tt MultiStepGridOperatorSpace} to the solver.  It will hold the vectors
of dofs $u^{n-1},\ldots,u^{n-m}$ as shared pointers in a queue and
automatically handle the management of the queue.

\section{Caching}

\subsection{Caching across Time Steps}

Consider time step $n$ and $n+1$: In step $n$ we solve
$\sum_{j=0}^pr_j(t^n,u^n)$ and evaluate $r_j(t^k,u^k)$ for $n<k\leq n-m$.  In
step $n+1$ we solve $\sum_{j=0}^pr_j(t^{n+1},u^{n+1})$ and evaluate
$r_j(t^k,u^k)$ for $n+1<k\leq n+1-m$.  In step $n+1$ we can reuse the results
of the evaluations $r_j(t^k,u^k)$ for $n<k<n-m$ from step $n$, i.e. $m-1$
evaluations of the operators $r_j$.  This is possible even for non-linear
operators.

\subsection{Caching for Affine Operators}

For affine operators we assume that $r_j$ can be split into a purely linear
part $\alpha_j$ and a constant part $\lambda_j$:
$r_j(t,u)=\alpha_j(t,u)+\lambda(t)$.  A local operator can advertise this by
giving its member constant {\tt hasPureLinearAlpha} the value {\tt true}.
This property has the nice consequence that we can write
$r_j(t,u)=J(r_j|_t)\cdot u+\lambda_j(t)$.  This allows us to evaluate the
Jacobian and $\lambda_j$ once for a given time $t$ and evaluate $r_j(t,u)$
many times with different $u$ without iterating over the grid.  This should be
particularly handy when solving.

Note that there may be operators which are in fact affine, but have {\tt
  alpha\_*()} methods which (partly) implement the affine shift.  These
operators can announce that they are affine by giving their member constant
{\tt isAffine} the value true.  For these operators, extraction of $r_j(t,0)$
and $J(r_j|_t)$ will be less efficient, the the values obtained can still be
cached.

It is an error for an operator to have {\tt hasPureLinearAlpha == true} but
{\tt isAffine == false}.

\subsection{Caching Stationary Parts of Affine Operators}

In the case of affine operators the Jacobian often does not change over time.
Temporal variability usually comes from boundary conditions, which are either
applied outside of the operator (Dirichlet) or contribute to the
$\lambda_j$-term (Neumann).  However, even the Jacobian may show changes over
time, for instance when material properties change or when it is coupled to a
different equation.  Even if changes do occur, they may occur apruptly,
seperated by long periods without changes, as in switching some inflow on or
off.

To deal with this we introduce the following method on temporal local
operators:
\begin{lstlisting}
void changeOccured(Time timeOfCachedValues, Time timeOfNeededValues,
                   bool &alphaChanged, bool &lambdaChanged) const;
\end{lstlisting}
By default they return true in both bools which inhibits caching.  This method
is called on a local operator before {\tt setTime(timeOfNeededValues)} is
called.  If {\tt changeOccured()} determines that nothing changed, the call to
{\tt setTime()} and subsequent evaluation calls may be skipped, and old values
used instead.

\subsection{Implementation}

The implementation consists of a caching {\tt
  CachedMultiStepGridoperatorSpace} and a companion {\tt MultiStepCache},
which the user may actually replace by his own class.  The interface for the
cache class is as follows:
\begin{lstlisting}
// Base exception for cache related errors
class CacheError : public Exception {};

// Exception thrown when a requested item is not in the cache
class NotInCache : public CacheError {};

// Exception thrown when a stored item is already in the cache
class AlreadyInCache : public CacheError {};

template<class VectorU, class VectorV, class Matrix, class Step = int>
struct MultiStepCacheInterface {
  // methods for the Jacobian of affine operators
  shared_ptr<const Matrix>
  getJacobian(std::size_t order, Step step) const;

  shared_ptr<const Matrix>
  getLatestJacobian(std::size_t order, Step &step) const;

  void setJacobian(Step step, std::size_t order,
                   const shared_ptr<const Matrix> &jacobian);

  // methods for the zero-residual of affine operators
  shared_ptr<const VectorV>
  getZeroResidual(std::size_t order, Step step) const;

  shared_ptr<const VectorV>
  getLatestZeroResidual(std::size_t order, Step &step) const;

  void setZeroResidual(Step step, std::size_t order,
                       const shared_ptr<const VectorV> &zeroResidual);

  // methods for non-linear caching
  shared_ptr<const VectorV>
  getResidualValue(std::size_t order, Step step) const;

  void setResidualValue(Step step, std::size_t order,
                        const shared_ptr<const VectorV> &residualValue);

  // methods to access old values of the unknowns
  shared_ptr<const VectorU> getUnknowns(Step step) const;
  void setUnknowns(Step step, const shared_ptr<const VectorU> &unknowns_);

  // methods to flush the cache
  void flushOldKeepLatestAffine(Step stepToKeep);
  void flushAll();
};
\end{lstlisting}
Note that it is perfectly legal for a cache implementation to silently refuse
to store a value in the cache.  A caching multi-step grid operator space shall
never assume it can extract a value from a cache it just stored there, and it
shall always be prepared to compute (and then store) any value it cannot find
in the cache.  The exception to this rule are of course the vectors of
unknowns.
\end{document}
