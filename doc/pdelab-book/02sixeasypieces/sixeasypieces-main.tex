\section{Abstract problem formulation}


\subsection{Continuous problem.}

Let $U$ and $V$ be linear function spaces. Then the residual based unconstrained
continuous problem reads 
\begin{equation}
\label{Eq:UnconstrainedContinuous}
u\in U \,: \qquad r(u,v) = 0 \qquad \forall v\in V .
\end{equation}
The mapping $r : U\times V \to \mathbb{K}$  is non-linear in its first argument but
linear in its second argument. 

In some cases a solution of problem \eqref{Eq:UnconstrainedContinuous}
exists but is not unique. Then additional constraints have to be
given, i.e. the problem is to be solved in a subspace. 
To that end let $\tilde{U}\subseteq U$ and
$\tilde{V}\subseteq V$. For a given function $w\in U$ we define the
affine subspace
\begin{equation*}
\text{``$w+\tilde{U}$''} = \{u\in U \,:\, u = w + \tilde{u}, \tilde{u}\in\tilde{U} \}.
\end{equation*}
Then the constrained continuous problem reads
\begin{equation}
u\in w+\tilde{U} \,: \qquad r(u,v) = 0 \qquad \forall v\in\tilde{V} .
\end{equation}

\begin{Exm}\label{exm:EllipticModelProblem}
Consider the elliptic model problem 
\begin{align*}
-\Delta u + a_0(x) u &= f &&\text{in $\Omega$},\\
u &= g && \text{on $\Gamma_D$},\\
-\nabla u\cdot\nu &= \mu && \text{on $\Gamma_N$},
\end{align*}
where $\partial\Omega = \Gamma_D \cup \Gamma_N$,
$\Gamma_D\cap\Gamma_N=\emptyset$ and $\nu$ denotes the outer unit
normal.

Appropriate function spaces for the weak formulation are $U = V =
H^1(\Omega)$ and the residual is 
\begin{equation*}
r(u,v) = \int\limits_{\Omega} \nabla u \cdot \nabla v + a_0 uv \,dx 
+ \int\limits_{\Gamma_N} \mu v \,ds - \int\limits_{\Omega} fv \,dx.
\end{equation*}
For $\Gamma_N=\partial\Omega$ and $a_0(x)\leq \alpha_0 > 0$ everywhere 
the unconstrained problem has a unique solution. 
For $\Gamma_N=\partial\Omega$, $a_0\equiv 0$ and
suitable $f$ the solution
is only defined up to a constant. The solution is unique in the
subspace
\begin{equation*}
\tilde{U} = \tilde{V} = \left\{ u\in H^1(\Omega) \,:\, 
\int_{\Omega} u \,dx = 0\right\} .
\end{equation*}
In the case of Dirichlet boundary conditions with
$\text{meas}\Gamma_D\neq 0$ the problem is unique in the subspace
\begin{equation*}
\tilde{U} = \tilde{V} = \left\{ u\in H^1(\Omega) \,:\, 
\text{$u = 0$ on $\Gamma_D$ in the sense of traces} \right\} .
\end{equation*}
\hfill$\square$
\end{Exm}

\subsection{Unconstrained Discrete Problem}

Let $U_h$ and $V_h$ be finite-dimensional linear function spaces
approximating $U$ and $V$ respectively. Note that we do not require that
$\tilde{U}_h\subseteq U_h$ and $\tilde{V}_h\subseteq V_h$.  
Then the unconstrained discrete problem reads:
\begin{equation}
u_h\in U_h \,: \qquad r_h(u_h,v) = 0 \qquad \forall v\in V_h .
\end{equation}
The residual $r_h : U_h \times V_h \to \mathbb{K}$ need not be
identical to $r$ in the case of non-conforming methods. 
We claim that a sufficiently large class of discretization schemes can be
formulated in this way.

Given a basis of the discrete function spaces we can reduce the
problem to its algebraic form. The basis is denoted by
\begin{align*}
U_h &= \text{span} \Phi_h, & \Phi_h &= \{\phi_j \,:\, j\in J\}, &
J\subset \mathbb{N}_0,\\
V_h &= \text{span} \Psi_h, & \Psi_h &= \{\psi_i \,:\, i\in I\}, &
I\subset \mathbb{N}_0,\\
\end{align*}
where $I, J$ are appropriate index sets. All functions can then be
represented with coefficients with respect to the basis:
\begin{align*}
\mathbf{U} &= \mathbb{K}^J, & \text{FE}_\Phi &: \mathbf{U} \to U_h, &
 \text{FE}_\Phi(\mathbf{u}) = \sum\limits_{j\in J} \mathbf{u}_j \phi_j,\\
\mathbf{V} &= \mathbb{K}^I, & \text{FE}_\Psi &: \mathbf{V} \to V_h, &
 \text{FE}_\Psi(\mathbf{v}) = \sum\limits_{i\in I} \mathbf{v}_i \psi_i.
\end{align*}
Then the unconstrained algebraic problem reads
\begin{align*}
\mathbf{u}\in\mathbf{U}\ &: & r_h(\text{FE}_\Phi(\mathbf{u}),
\text{FE}_\Psi(\mathbf{v})) &= 0 &&\mathbf{v}\in\mathbf{V}\\
&\Leftrightarrow & r_h(\text{FE}_\Phi(\mathbf{u}),
\psi_i) &= 0 &&i\in I.
\end{align*}
By setting 
\begin{align*}
\mathcal{R} &: \tilde{\mathbf{U}}\to\tilde{\mathbf{V}}, &
\left(\mathcal{R}(\mathbf{u})\right)_i &= r_h(\text{FE}_\Phi(\mathbf{u}),\psi_i),
\end{align*}
the unconstrained discrete problem can be written as a nonlinear
algebraic system 
\begin{equation}
\mathbf{u}\in\mathbf{U} \,: \qquad 
\mathcal{R}(\mathbf{u}) = \mathbf{0}.
\end{equation}
The solution of this system might be computed by Newton's method which
reads in that case
\begin{equation}
\mathbf{u}^{k+1} = \mathbf{u}^{k} -
\left(\nabla\mathcal{R}(\mathbf{u}^{k})\right)^{-1}\mathcal{R}(\mathbf{u}^{k}) .
\end{equation}
Here $\nabla\mathcal{R}(\mathbf{u}^{k})$ denotes the Jacobian of the
map $\mathcal{R}$ which is defined as 
\begin{equation*}
\left(\nabla\mathcal{R}(\mathbf{u}^{k})\right)_{i,j} = \frac{\partial
  r_h(\text{FE}_\Phi(\mathbf{u}^k),\psi_i)}{\partial \mathbf{u}_j} .
\end{equation*}

\subsection{Constrained Discrete Problem}

As in the continuous case the solution of the unconstrained discrete
problem might not be unique. To achieve uniqueness the problem needs
to be constrained to subspaces $\tilde{U}_h \subset U_h$, $\tilde{V}_h
\subset V_h$. Then the constrained discrete problem reads:
\begin{equation}\label{eq:ConstrainedDiscreteProblem1}
u_h\in w+\tilde{U}_h \,: \qquad r_h(u_h,v) = 0 \qquad \forall v\in \tilde{V}_h .
\end{equation}

\begin{Exm}\label{exm:ConstrainedDiskreteSpaces}
We extend example \ref{exm:EllipticModelProblem}. Let $\mathcal{T}(\Omega)$
denote a triangulation of the domain.
\begin{enumerate}[a)]
\item Assume that $\mathcal{T}(\Omega)$ is conforming. Then the space
  of piecewise linear finite elements is given by
\begin{equation*}
U_h = V_h = \left\{ u\in C^0(\Omega) \,:\, \text{$u|_t$ is linear on each
  $t\in\mathcal{T}(\Omega)$} \right\} .
\end{equation*}
In the case of pure Neumann boundaries we then take, as in the
continuous case:
\begin{equation*}
\tilde{U}_h = \tilde{V}_h = \left\{ u\in U_h \,:\, 
\int_{\Omega} u \,dx = 0\right\} .
\end{equation*}
\item In the case of Dirichlet boundary condition we simply take
\begin{equation*}
\tilde{U}_h = \tilde{V}_h = \left\{ u\in U_h \,:\, 
\text{$u = 0$ on $\Gamma_D$} \right\} .
\end{equation*}
\item As a third case we consider hanging nodes. In that case
  $\mathcal{T}(\Omega)$ is a non-conforming triangulation and we set
\begin{equation*}
U_h = V_h = \left\{ u\in L_2(\Omega) \,:\, \text{$u|_t$ is linear on each
  $t\in\mathcal{T}(\Omega)$} \right\} .
\end{equation*}
The constrained spaces then are $\tilde{U}_h = \tilde{V}_h = U_h \cap
C^0(\Omega)$. \hfill$\square$
\end{enumerate}
\end{Exm} 

\paragraph{Basis Transformation.} For the algebraic formulation of the
discrete constrained problem we chose new sets of basis functions
$\tilde{\Phi}_h$, $\tilde{\Psi}_h$ generating the spaces $U_h$ and
$V_h$. The finite element isomorphism then is
\begin{align*}
\text{FE}_{\tilde\Phi} &: \mathbf{U} \to U_h, &
 \text{FE}_{\tilde\Phi}(\tilde{\mathbf{u}}) &= \sum\limits_{j\in J}
 \tilde{\mathbf{u}}_j \tilde\phi_j,\\ 
\text{FE}_{\tilde\Psi} &: \mathbf{V} \to V_h, &
 \text{FE}_{\tilde\Psi}(\tilde{\mathbf{v}}) &= \sum\limits_{i\in I}
 \tilde{\mathbf{v}}_i \tilde\psi_i. 
\end{align*}

The new basis functions are chosen in such a way that for
$\tilde{J}\subset J$ and $\tilde{I}\subset I$ we have
\begin{align*}
\tilde{U}_h &= \text{span} \left\{\tilde\phi_j \,:\, j\in\tilde{J}\right\},  &
\tilde{V}_h &= \text{span} \left\{\tilde\psi_i \,:\, i\in\tilde{I}\right\}.
\end{align*}

Morevover, the new basis functions can be represented in terms of the
basis functions of the unconstrained problem with the help of the coefficient matrices
$\mathbf{S}\in\mathbb{K}^{J\times J}$ and
$\mathbf{T}\in\mathbb{K}^{I\times I}$:
\begin{align*}
\tilde\phi_j &= \sum_{l\in J} \mathbf{S}_{j,l} \phi_l, &
\tilde\psi_i &= \sum_{k\in I} \mathbf{T}_{i,k} \phi_k.
\end{align*}
Thus one can verify 
\begin{align*}
\text{FE}_{\tilde\Phi}(\tilde{\mathbf{u}}) &=
\text{FE}_{\Phi}(\mathbf{S}^T\tilde{\mathbf{u}}), &
\text{FE}_{\tilde\Psi}(\tilde{\mathbf{v}}) &=
\text{FE}_{\Psi}(\mathbf{T}^T\tilde{\mathbf{v}}).
\end{align*}
The matrix $\mathbf{S}^T$ transforms a coefficient vector with respect
to the new basis $\tilde\Phi_h$ into a coefficient vector with respect
to the old basis $\Phi_h$. Similarly, $\mathbf{T}^T$ provides the
basis transformation from $\tilde\Psi_h$ to $\Psi_h$.

\begin{Exm}\label{exm:BasisTransformation}
We extend the example \ref{exm:ConstrainedDiskreteSpaces}.
Consider $J = \{0,\ldots,n-1\}$ and let $\Phi_h$ denote the standard
  nodal basis functions.
\begin{enumerate}[a)]
\item In the pure Neumann case we set $\tilde{J} = J \setminus
  \{n-1\}$ and 
\begin{equation*}
\tilde\phi_j = \left\{\begin{array}{ll}
\phi_j - \phi_{n-1} & j \in \tilde{J}\\
\phi_0 + \ldots + \phi_{n-1} & \text{else}
\end{array}\right. .
\end{equation*}
\item Dirichlet case. Let $\tilde{J}\subset J$ be the indices
  corresponding to nodes that are \textit{not} on the Dirichlet
  boundary and set $\tilde\phi_j = \phi_j$ for all $j\in J$. This
  means $\tilde\Phi_h = \tilde\Phi$ in that case.
\item Hanging node case. Denote by $\tilde{J}$ the indices of the
  nodes that are \textit{not} hanging nodes. Consequently
  $J\setminus\tilde{J}$ is the index set of the hanging nodes. Then we
  have 
\begin{equation*}
\tilde\phi_j = \left\{\begin{array}{ll}
\phi_j + \sum_{l\in J\setminus\tilde{J}} \mathbf{S}_{j,l} \phi_{l} & j \in \tilde{J}\\
\phi_j & \text{else}
\end{array}\right. .
\end{equation*}
The coefficients $\mathbf{S}_{j,l}$, $j \in \tilde{J}$, $l\in
J\setminus\tilde{J}$ are chosen in such a way that $\phi_j$ is
continuous. This is very similar to a hierarchical basis
representation. \hfill$\square$
\end{enumerate}
\end{Exm}

\paragraph{Orthogonal Projection.}
We introduce a projection $Q_h : U_h \to \tilde{U}_h$ in the following way:
\begin{equation*}
Q_h \tilde\phi_j = \left\{\begin{array}{ll}
\tilde\phi_j & j\in\tilde{J},\\
0 & \text{else}.
\end{array}\right.
\end{equation*}
One easily verifies the following properties:
\begin{enumerate}[i)]
\item $\text{im}(Q_h) = \tilde{U}_h$,
\item $Q_h^2 = Q_h$,
\item $\text{ker}(Q_h) = \text{im}(I-Q_h)$.
\item $U_h = \text{im}(Q_h) \oplus \text{ker}(Q_h)$.
\end{enumerate}

\begin{Ass} In order to simplify things we assume that the affine shift
  fulfills $w\in\text{ker}(Q_h)$.
\hfill$\square$
\end{Ass}

Then problem \eqref{eq:ConstrainedDiscreteProblem1} can be rewritten
as
\begin{subequations}\label{eq:ConstrainedDiscreteProblem2}
\begin{align}
u\in U_h \quad : && r_h( u_h , v ) &= 0 \quad \forall v \in
\tilde{V}_h, \\
&& (I-Q_h) u_h &= w .
\end{align}
\end{subequations}

The advantage of this formulation is that it fixes a unique element in
the unconstrained space $U_h$ by stating an additional equation.
Note that the projection $Q_h$ need not be formed in the
implementation. It is only introduced to derive the algebraic
formulation. 

The effect of the projection in terms of coefficients is easy to
compute. By defining 
\begin{equation*}
\mathbf{Q} : \mathbf{U} \to \mathbf{U}, \qquad
(\mathbf{Q}\mathbf{u})_j = \left\{\begin{array}{ll}
(\mathbf{u})_j & j\in \tilde{J},\\
0 & \text{else},
\end{array}\right.
\end{equation*}
we have
\begin{align*}
Q_h \text{FE}_{\tilde\Phi}(\tilde{\mathbf{u}}) &=
Q_h \left(\sum_{j\in J} (\mathbf{u})_j \tilde\phi_j \right) = 
\sum_{j\in J} (\mathbf{u})_j Q_h \tilde\phi_j = 
\sum_{j\in \tilde{J}} (\mathbf{u})_j Q_h \tilde\phi_j \\
&= \text{FE}_{\tilde\Phi}(\mathbf{Q} \tilde{\mathbf{u}}).
\end{align*}
Accordingly we have $(I-Q_h)
\text{FE}_{\tilde\Phi}(\tilde{\mathbf{u}}) = 
\text{FE}_{\tilde\Phi}((\mathbf{I}-\mathbf{Q})\tilde{\mathbf{u}})$.

Assuming $\mathbf{Q}\tilde{\mathbf{w}}=\mathbf{0}$ we can rewrite
\eqref{eq:ConstrainedDiscreteProblem2} in algebraic form:
\begin{subequations}
\begin{align*}
\tilde{\mathbf{u}}\in\mathbf{U} \quad : &&
r_h(\text{FE}_{\tilde\Phi}(\tilde{\mathbf{u}}),\tilde\psi_i) &= 0 \quad \forall i \in
\tilde{I}, \\
&& (\mathbf{I}-\mathbf{Q})\tilde{\mathbf{u}} &= \tilde{\mathbf{w}} .
\end{align*}
\end{subequations}
Note that the solution is a vector of coefficients with respect to the
basis $\tilde\Phi_h$. Using the basis transformation we obtain
\begin{equation*}
r_h(\text{FE}_{\tilde\Phi}(\tilde{\mathbf{u}}),\tilde\psi_i) = 
r_h\left(\text{FE}_{\Phi}(\mathbf{S}^T\tilde{\mathbf{u}}),\sum_{k\in
  I}\mathbf{T}_{i,k}\psi_k\right)
= \sum_{k\in I} \mathbf{T}_{i,k}
\, r_h\left(\text{FE}_{\Phi}(\mathbf{S}^T\tilde{\mathbf{u}}),\psi_k\right) .
\end{equation*}
Introducing the $\mathbb{K}^{\tilde{I}\times I}$ submatrix of
$\mathbf{T}$
\begin{equation*}
\left(\tilde{\mathbf{T}}\right)_{i,k} = \left(\mathbf{T}\right)_{i,k}
\qquad \forall (i,k) \in \tilde{I}\times I ,
\end{equation*}
the algebraic problem can be written as
\begin{subequations}
\begin{align}\label{eq:ConstrainedDiscreteProblem3}
\tilde{\mathbf{u}}\in\mathbf{U} \quad : &&
\tilde{\mathbf{T}} \mathcal{R}(\mathbf{S}^T\tilde{\mathbf{u}}) = \mathbf{0} \\
&& (\mathbf{I}-\mathbf{Q})\tilde{\mathbf{u}} &= \tilde{\mathbf{w}} .
\end{align}
\end{subequations}

Solving this using
Newon's method we seek a new iterate $\tilde{\mathbf{u}}^{k+1} =
\tilde{\mathbf{u}}^{k} + \tilde{\mathbf{z}}^{k}$ where
\begin{align*}
\tilde{\mathbf{T}}\mathcal{R}(\mathbf{S}^T\tilde{\mathbf{u}}^k)
 + \tilde{\mathbf{T}}
 \nabla\mathcal{R}(\mathbf{S}^T\tilde{\mathbf{u}}^k)\mathbf{S}^T\tilde{\mathbf{z}}^{k}
 &= \mathbf{0} \\
(\mathbf{I}-\mathbf{Q})(\tilde{\mathbf{u}}^{k} +
 \tilde{\mathbf{z}}^{k}) &= \tilde{\mathbf{w}}  .
\end{align*}
Assuming that the current iterate fulfills
$(\mathbf{I}-\mathbf{Q})\tilde{\mathbf{u}}^k = \tilde{\mathbf{w}}$ and
rearranging the first equation we obtain
\begin{align*}
\tilde{\mathbf{T}}
 \nabla\mathcal{R}(\mathbf{S}^T\tilde{\mathbf{u}}^k)\mathbf{S}^T\tilde{\mathbf{z}}^{k}
 &= - \tilde{\mathbf{T}}\mathcal{R}(\mathbf{S}^T\tilde{\mathbf{u}}^k) \\
(\mathbf{I}-\mathbf{Q}) \tilde{\mathbf{z}}^{k} &= \mathbf{0} .
\end{align*}
In order to solve this system we introduce
\begin{align*}
\mathbf{U}_{\tilde{J}} &= \mathbb{K}^{\tilde{J}}, &
\mathbf{R}_{\tilde{J}}
  &:  \mathbf{U} \to \mathbf{U}_{\tilde{J}}, &
  (\mathbf{R}_{\tilde{J}}\mathbf{u})_j &= (\mathbf{u})_j \quad \forall
j\in\tilde{J},\\ 
\mathbf{U}_{J\setminus\tilde{J}} &= \mathbb{K}^{J\setminus\tilde{J}}, &
\mathbf{R}_{J\setminus\tilde{J}}
  &:  \mathbf{U} \to \mathbf{U}_{J\setminus\tilde{J}}, &
  (\mathbf{R}_{J\setminus\tilde{J}}\mathbf{u})_j &= (\mathbf{u})_j
\quad \forall j\in J\setminus\tilde{J}
\end{align*}
and split the coefficient vector into two parts
$\mathbf{u}_{\tilde{J}} = \mathbf{R}_{\tilde{J}} \mathbf{u}$
and $\mathbf{u}_{J\setminus\tilde{J}} = \mathbf{R}_{J\setminus\tilde{J}} \mathbf{u}$. 
Finally introducing the $\mathbb{K}^{\tilde{J}\times J}$ submatrix of
$\mathbf{S}$
\begin{equation*}
\left(\tilde{\mathbf{S}}\right)_{j,l} = \left(\mathbf{T}\right)_{j,l}
\qquad \forall (j,l) \in \tilde{J}\times J ,
\end{equation*}
the linear system for the update can be written as
\begin{align*}
\tilde{\mathbf{T}}
 \nabla\mathcal{R}(\mathbf{S}^T\tilde{\mathbf{u}}^k)
 \tilde{\mathbf{S}}^T\tilde{\mathbf{z}}^{k}_{\tilde{J}}  
 &= - \tilde{\mathbf{T}}\mathcal{R}(\mathbf{S}^T\tilde{\mathbf{u}}^k) \\
\tilde{\mathbf{z}}^{k}_{J\setminus\tilde{J}} &= \mathbf{0} .
\end{align*}

The complete Newton' scheme for the solution of the algebraic
constrained problem then reads as follows. Given an initial guess with
$(\mathbf{I}-\mathbf{Q})\tilde{\mathbf{u}}^0 = \tilde{\mathbf{w}}$ iterate
\begin{equation*}
\tilde{\mathbf{u}}^{k+1} = \tilde{\mathbf{u}}^{k} - 
\mathbf{R}_{\tilde{J}}^T \left (\tilde{\mathbf{T}}
 \nabla\mathcal{R}(\mathbf{S}^T\tilde{\mathbf{u}}^k)
 \tilde{\mathbf{S}}^T \right)^{-1}
 \tilde{\mathbf{T}}\mathcal{R}(\mathbf{S}^T\tilde{\mathbf{u}}^k) .
\end{equation*}
Applying the basis transformation yields
\begin{equation*}
\mathbf{S}^T\tilde{\mathbf{u}}^{k+1} = \mathbf{S}^T\tilde{\mathbf{u}}^{k} - 
\mathbf{S}^T\mathbf{R}_{\tilde{J}}^T \left (\tilde{\mathbf{T}}
 \nabla\mathcal{R}(\mathbf{S}^T\tilde{\mathbf{u}}^k)
 \tilde{\mathbf{S}}^T \right)^{-1}
 \tilde{\mathbf{T}}\mathcal{R}(\mathbf{S}^T\tilde{\mathbf{u}}^k) .
\end{equation*}

In algorithmic form this can be written as follows. Given an initial
iterate $\mathbf{u}^0$ with $Q_h \text{FE}_{\Phi}(\mathbf{u}^0) = 0$
do for $k=0,1,\ldots$ :
\begin{enumerate}[1)]
\item Compute residual $\mathbf{r}_{\tilde{J}} =
  \tilde{\mathbf{T}}\mathcal{R}(\mathbf{u}^k)$. 
\item Solve the linear system 
\begin{equation*}
\left(\begin{array}{cc}
\tilde{\mathbf{T}}
 \nabla\mathcal{R}(\mathbf{u}^k)
 \tilde{\mathbf{S}}^T & \mathbf{0}\\
\mathbf{0} & \mathbf{I}
\end{array}\right)
\tilde{\mathbf{z}} = \left(\begin{array}{cc}
\mathbf{r}_{\tilde{J}}\\
\mathbf{0}
\end{array}\right) .
\end{equation*}
\item Update using basis transformation: $\mathbf{u}^{k+1} =
  \mathbf{u}^{k} + \mathbf{S}^T \tilde{\mathbf{z}}$.
\end{enumerate}
